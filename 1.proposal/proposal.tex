\documentclass[sigchi-a, authorversion]{acmart}
\usepackage{booktabs} % For formal tables
\usepackage{ccicons}  % For Creative Commons citation icons
\usepackage{url}
\usepackage{hyperref}


% Information
\acmConference[SPWAL SS17]{Seminar und Praktikum "Wissenschaftliches Arbeiten und Lehren" Sommersemester 2017}{\today}{Munich, Germany} 

\begin{document}
\title{A system to regulate E-waste export in the EU.}

% Author
\author{Anita Baier}
\affiliation{\institution{Ludwigs-Maximilian-Universit\"at}
  \city{Munich} \country{Germany}}
\email{anita.baier@gmx.de} 

\author{Zhe Li}
\affiliation{\institution{Ludwigs-Maximilian-Universit\"at}
  \city{Munich} \country{Germany}}
\email{li.zhe@campus.lmu.de} 

\author{Jonas Mattes}
\affiliation{\institution{Ludwigs-Maximilian-Universit\"at}
  \city{Munich} \country{Germany}}
\email{j.mattes@campus.lmu.de} 

\author{Bruno M\"uller}
\affiliation{\institution{Ludwigs-Maximilian-Universit\"at}
  \city{Munich} \country{Germany}}
\email{muellerbr@cip.ifi.lmu.de} 

\author{An Tien Go}
\affiliation{\institution{Ludwigs-Maximilian-Universit\"at}
  \city{Munich} \country{Germany}}
\email{antiengo@campus.lmu.de}

\author{Changkum Ou}
\affiliation{\institution{Ludwigs-Maximilian-Universit\"at}
  \city{Munich} \country{Germany}}
\email{hi@changkun.us} 

\author{Guoliang Xue}
\affiliation{\institution{Ludwigs-Maximilian-Universit\"at}
  \city{Munich} \country{Germany}}
\email{guoliang.xue@campus.lmu.de} 

\ccsdesc[500]{Hardware~Power and energy}
\ccsdesc[300]{Hardware~Impact on the environment}


\begin{abstract}

  In todays society electronics become a more and more crucial part of our lives. 
  The demand for electronic devices increases and so does the amount of waste we produce with them. 
  E-waste is a term used for old electronic devices that are of no use anymore and thus become scrap. 
  This causes serious pollution to our surroundings and other countries since most of the e-waste is 
  being exported to third world countries.
  
  Our goal is to find a solution to this pollution problem. Our plan is to create a trading system similar 
  to the CO2 trading system of the European Union. Countries are allowed to only export a certain amount of e-waste. 
  This amount can be increased if they either buy allowances from other countries or if they invest in recycling of e-waste. 
  The global e-waste export limits will be lowered year by year, thus lowering global pollution created by no longer
  used electronics.  
  
\end{abstract}

\keywords{E-waste; Fee-Upon-Disposal Regulation; recycling}

\maketitle

% \begin{sidebar}
%   \textbf{Good Utilization of the Side Bar} 
  
%   \textbf{Preparation:} Do not change the margin
%   dimensions and do not flow the margin text to the
%   next page. 
  
%   \textbf{Materials:} The margin box must not intrude
%   or overflow into the header or the footer, or the gutter space
%   between the margin paragraph and the main left column. 
  
%   \textbf{Images \& Figures:} Practically anything
%   can be put in the margin if it fits. Use the
%   \texttt{{\textbackslash}marginparwidth} constant to set the
%   width of the figure, table, minipage, or whatever you are trying
%   to fit in this skinny space.

%   \caption{This is the optional caption}
%   \label{bar:sidebar}
% \end{sidebar}

%\begin{figure}
%  \includegraphics[width=\marginparwidth]{sigchi-logo}
%  \caption{Insert a caption below each figure.}
%  \label{fig:sample}
%\end{figure}


\section{Introduction}
E-waste is a growing problem, especially in industrialized countries. One of the applicated solutions
is to ship this e-waste to third world countries. However, this cannot be the longterm solution. As such
we propose an e-waste allowance system. With this, only a limited amount of e-waste can be shipped outside, 
which will be reduced yearly.

\section{Related Work}

%\subsection{Jonas Mattes}
Robinson \cite{Robinson2009} pointed out that most of the E-waste isn't even getting collected and just thrown into the household waste. 80\% of the E-waste which got collected is then getting exported in poor countries. The recycling in these countries is problematic because E-waste contains lots of environmental contaminants and the facilities doesn't take proper care of this. This is why these contaminants are found around these premises. E-waste has already caused a "`considerable environmental degradation"'\cite{Robinson2009} in these countries. Also the workers are suffering from health problems because barely protected against the dangerous fluids and gasses. 

%\subsection{Anita Baier}
According to the current european WEEE-directive, manufacturers, sellers and distributors need to 
provide a return point for electronical and electrical devices. The aim is amongst others the reinforcement 
of recycling upon responsibility of the producer, which are also in charge of bearing the costs, while the 
end consumer has the responsibility of proper waste separation \cite{EURLEX}. Specifically for smartphones, 
the German Goverment rejects a deposit at the expense of the final consumer on the national implementation \cite{BMUB}. 
There are also existing several non-profit projects, which accept mobile phones in order to reuse and recycle them \cite{NABU, DUH}. 

%\subsection{Bruno M\"uller}
\begin{quotation}
  "\textit{Cumulatively, about 500 million PCs reached the end of their service lives between 1994 and 2003. 500 million PCs contain approximately 2,872,000 t of plastics, 718,000 t of lead, 1363 t of cadmium and 287 t of mercury}" \cite{global-perspective}\\
  
\end{quotation}

According to \cite{global-perspective} this already huge amount of e-waste is going to increase even further as electronics keep advancing and the need for new electronics keeps increasing. Exporting e-waste to poor countries would make sense for first world countries according to Larry Summers (back in 1991) since third world countries don't have an industry that already pollutes their air, water and ground so heavily, so they can deal with that problem more easily. Plus since mortality rates are already so high in these areas, the added pollution would not affect these countries that much.\\
This thinking started to change with the Basel Convention in 1989. It limits how much e-waste can be moved to what parts of the world, trying to save the environment and also trying to push the companies towards recycling. \\

Large household appliances and IT and telecommunications equipment made up three quarters of all e-waste back in 2002. This e-waste was mainly generated by countries of the OECD. Overall numbers are going to keep on increasing.\\

Although it is hard to track down how and where e-waste is going, there have to be put some restrictions into action on where e-waste is going. To solve the problem of this steadily increasing waste, recycling should become more important and more commonly used. But even for this solution, there are environmental hazards that come along with recycling.\\
To prevent e-waste from even becoming a problem, products should also be designed in such a way that they thrown away in a relatively short time span. 

%\subsection{Changkun Ou}
From the regulation aspect, Plambeck and Wang  \cite{plambeck2009}  investigates the influence of how e-waste regulation types (Fee Upon Sale and Fee Upon Disposal) and market structure (Duopoly and Monopoly) affects related four aspects of new production indtroduction: the quantify of e-waste, manufacturers' profits, consumer surplus and social welfare. 
As instance, they considered the California's Advanced Recovery Fee (ARF) which is a fee-upon-sale strategy that forces customer pay for the collection and recycling of all used electronics when they buy an new product. For analysis, their models explained the fee-upon-disposal extended producer responsibility motivate design for recyclability which is benefits for consumers but fail to reduce the frequency of new products launching.
In consequences, optimally induces electronics manufactures to both slow research/development and design benign products is the further challenge for our future form invention of e-waste regulation.

%\subsection{Zhe Li}
The EU emissions trading system (EU ETS)\cite{EU-ETS}, also known as the European Union Emissions Trading Scheme, 
was the first large greenhouse gas emissions trading scheme in the world, and now it's still the biggest one. 
The EU ETS works on the 'cap and trade' principle. A cap is set on the total amount of certain greenhouse gases 
that can be emitted by installations covered by the system. The total emissions will fall if the cap is reduced 
over time. 

Under the 'cap and trade' principle, a maximum (cap) is set on the total amount of greenhouse gases that can be 
emitted by all participating installations. 'Allowances' for emissions are then auctioned off or allocated for 
free, and can subsequently be traded. Companies receive or buy emission allowances which they can trade with one 
another as needed. They can also buy limited amounts of international credits from emission-saving projects around 
the world. Trading brings flexibility that ensures emissions are cut where it costs least to do so. A robust carbon 
price also promotes investment in clean, low-carbon technologies.

Installations must monitor and report their CO2 emissions, ensuring they hand in enough allowances to the authorities 
to cover their emissions. If emission exceeds what is permitted by its allowances, an installation must purchase 
allowances from others.

The system covers the following sectors and gases with the focus on emissions that can be measured, reported 
and verified with a high level of accuracy: carbon dioxide (CO2), nitrous oxide (N2O), perfluorocarbons (PFCs).

According to the European Commission, in 2010 greenhouse gas emissions from big emitters covered by the EU ETS 
had decreased by an average of more than 17,000 tonnes per installation from 2005, a decrease of more than 8 percent since 2005.

%\subsection{An}
One way to reduce e-waste is by successfully recycling outdated devices. There have been many proposals for this.
For example, Kahhat et al. introduced a system similar to the ``bottle bill", which enforces recycling. A consumer has to pay
an additional deposit when buying new devices, which are returned on successful return \cite{kahhat2008exploring}.

%\subsection{Guoliang Xue}
\documentclass[dvips,12pt]{article}

\usepackage[pdftex]{graphicx}
\usepackage{url}

% These are additional packages for "pdflatex", graphics, and to include
% hyperlinks inside a document.

\setlength{\oddsidemargin}{0.25in}
\setlength{\textwidth}{6.5in}
\setlength{\topmargin}{0in}
\setlength{\textheight}{8.5in}

% These force using more of the margins that is the default style

\begin{document}
	
	\title{A Proposal for E-WASTE Management}
	\author{Guoliang Xue}
	\date{\today}
	\maketitle
	
	\section*{Key Points}
	
	People are facing a hazardous problem in the modern era, more specifically saying, there are 20-50 million tons of e-waste being produced per year around the world, and this trend is still keeping going up. 
	
	The e-waste originally appeared in the developed countries, but now expanded to the developing countries, even less developed countries, which brings them both advantage and disadvantage.
	
	The reason why e-waste has such contradictory traits mainly lies on the two aspects of e-waste, which are supposed to be treated respectively:
	\begin{itemize}
		\item Poisonous parts in the e-waste.(disadvantages)
		\item Valuable parts in the e-waste.(advantages)
	\end{itemize}
   Therefore, how to maximize the advantages, meanwhile minimize the disadvantage becomes a priority for those suffering countries.
   
   Traditional ways to deal with e-waste will consist of four procedures: Recycling, Landfills, Composition of e-waste, and Security issues.
   The paper also provides several approaches: Life Cycle Assessment, Material Flow Analysis, Multi-Criteria Analysis , Extended Producer Responsibility, Health Implications.
   
   A good management system should apply above approaches into procedures. How can we innovate over the traditional method and put forward a more scientific and efficient way could be a interesting proposal for our final project.
   
   
   
   \begin{thebibliography}{99}
   	
   	\bibitem{gonzalez2012} 
   	Munam Ali Shah,	
   	Rakhshanda Batool,	
   	{An Overview of Electronic Waste Management, Practices and Impending Challenges},
   	International Journal of Computer Applications (0975 – 8887), {\bf Volume 125}, No.2 (2015).
   	
   \end{thebibliography}
   
	
\end{document}

\section{Justification}

Many solution to the e-waste problems are extreme (as seen in the related work section). However, it is important
for a successful transfer from old plans to efficient e-waste plans to make small adjustments first. 
For this reason, we introduce an e-waste allowance. This will force countries to slowly but surely reduce their e-waste production.

\section{Evaluation}

In order to check the vadility of this approach, we will introduce this system first for specific device. 
For example, only TVs should be limited in exporting. With the results of the testing phase, we will 
re-evaluate our system.

\section{Research Plan}

Our system will be similar to the C02 system of the EU. As such, we will analyze the C02 plan and re-adjust 
it to our needs. For this, these important aspects have to be considered: the measurement of e-waste, 
the distribution of e-waste-shipping-certificates (similar to the C02-certificates) and how many certificates 
should be available. 
In the first progress report, we will present a draft-system for this. In the mid term syncrhonisation we expect
to have the system more fleshed out. In the second report we expect to have the system completely worked out.
In the final stage, we will expand our research question in other ways. For example, with our proposed idea no clear
progress can be measured (when considering everything), as only returned e-waste is monitored. One solution to this would
be to research how e-waste could be tracked. 

\bibliography{bibfile}
\bibliographystyle{bibformat}

\end{document}
