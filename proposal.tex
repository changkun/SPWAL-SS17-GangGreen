\documentclass[sigchi-a, authorversion]{acmart}
\usepackage{booktabs} % For formal tables
\usepackage{ccicons}  % For Creative Commons citation icons
\usepackage{url}
\usepackage{hyperref}

% Copyright
%\setcopyright{none}
\setcopyright{acmcopyright}
%\setcopyright{acmlicensed}
%\setcopyright{rightsretained}
%\setcopyright{usgov}
%\setcopyright{usgovmixed}
%\setcopyright{cagov}
%\setcopyright{cagovmixed}


% DOI
\acmDOI{10.475/123_4}

% ISBN
\acmISBN{123-4567-24-567/08/06}

%Conference
\acmConference[WOODSTOCK'97]{ACM Woodstock conference}{July 1997}{El
  Paso, Texas USA} 
\acmYear{1997}
\copyrightyear{2016}

\acmPrice{15.00}

%\acmBadgeL[http://ctuning.org/ae/ppopp2016.html]{ae-logo}
\acmBadgeR[http://ctuning.org/ae/ppopp2016.html]{ae-logo}

\begin{document}
\title{A cool system to make E-waste great again.}

% enter your name and credentials here in order of alphabet
\author{Anita Baier}
\affiliation{\institution{University of Umbhali}
  \city{Pretoria} \country{South Africa}}
\email{author7@umbhaliu.ac.za}

\author{Zhe Li}
\affiliation{\institution{University of Umbhali}
  \city{Pretoria} \country{South Africa}}
\email{author7@umbhaliu.ac.za}

\author{Jonas Mattes}
\affiliation{\institution{University of Umbhali}
  \city{Pretoria} \country{South Africa}}
\email{author7@umbhaliu.ac.za}

\author{Bruno M\"uller}
\affiliation{\institution{University of Umbhali}
  \city{Pretoria} \country{South Africa}}
\email{author7@umbhaliu.ac.za}

\author{An}
\affiliation{\institution{University of Umbhali}
  \city{Pretoria} \country{South Africa}}
\email{author7@umbhaliu.ac.za}

\author{Changkum Ou}
\affiliation{\institution{University of Umbhali}
  \city{Pretoria} \country{South Africa}}
\email{author7@umbhaliu.ac.za} 

\author{Guoliang Xue}
\affiliation{\institution{University of Umbhali}
  \city{Pretoria} \country{South Africa}}
\email{author7@umbhaliu.ac.za} 

% The default list of authors is too long for headers}
\renewcommand{\shortauthors}{F. Author et al.}


%
% The code below should be generated by the tool at
% http://dl.acm.org/ccs.cfm
% Please copy and paste the code instead of the example below. 
%
\begin{CCSXML}
<ccs2012>
 <concept>
  <concept_id>10010520.10010553.10010562</concept_id>
  <concept_desc>Computer systems organization~Embedded systems</concept_desc>
  <concept_significance>500</concept_significance>
 </concept>
 <concept>
  <concept_id>10010520.10010575.10010755</concept_id>
  <concept_desc>Computer systems organization~Redundancy</concept_desc>
  <concept_significance>300</concept_significance>
 </concept>
 <concept>
  <concept_id>10010520.10010553.10010554</concept_id>
  <concept_desc>Computer systems organization~Robotics</concept_desc>
  <concept_significance>100</concept_significance>
 </concept>
 <concept>
  <concept_id>10003033.10003083.10003095</concept_id>
  <concept_desc>Networks~Network reliability</concept_desc>
  <concept_significance>100</concept_significance>
 </concept>
</ccs2012>  
\end{CCSXML}

\ccsdesc[500]{Computer systems organization~Embedded systems}
\ccsdesc[300]{Computer systems organization~Redundancy}
\ccsdesc{Computer systems organization~Robotics}
\ccsdesc[100]{Networks~Network reliability}


\begin{abstract}
  UPDATED---\today. This sample paper describes the formatting
  requirements for SIGCHI Extended Abstract Format, and this sample
  file offers recommendations on writing for the worldwide SIGCHI
  readership. Please review this document even if you have submitted
  to SIGCHI conferences before, as some format details have changed
  relative to previous years. Abstracts should be about 150
  words. Required.
\end{abstract}


\keywords{Authors' choice; of terms; separated; by
  semicolons; include commas, within terms only; required.}



\maketitle

\begin{sidebar}
  \textbf{Good Utilization of the Side Bar} 
  
  \textbf{Preparation:} Do not change the margin
  dimensions and do not flow the margin text to the
  next page. 
  
  \textbf{Materials:} The margin box must not intrude
  or overflow into the header or the footer, or the gutter space
  between the margin paragraph and the main left column. 
  
  \textbf{Images \& Figures:} Practically anything
  can be put in the margin if it fits. Use the
  \texttt{{\textbackslash}marginparwidth} constant to set the
  width of the figure, table, minipage, or whatever you are trying
  to fit in this skinny space.

  \caption{This is the optional caption}
  \label{bar:sidebar}
\end{sidebar}

\begin{figure}
  \includegraphics[width=\marginparwidth]{sigchi-logo}
  \caption{Insert a caption below each figure.}
  \label{fig:sample}
\end{figure}


\section{Introduction}
This format is to be used for submissions that are published in the
conference publications. We wish to give this volume a consistent,
high-quality appearance. We therefore ask that authors follow some
simple guidelines. In essence, you should format your paper exactly
like this document. The easiest way to do this is to replace the
content with your own material.


\section{ACM Copyrights \& Permission}
Accepted extended abstracts and papers will be distributed in the
Conference Publications. They will also be placed in the ACM Digital
Library, where they will remain accessible to thousands of researchers
and practitioners worldwide. To view the ACM's copyright and
permissions policy, see:
\url{http://www.acm.org/publications/policies/copyright_policy}.


\section{Related Work}
We don't work on the main file. Everybody includes his own .tex file :)
Robinson \cite{Robinson2009} pointed out that most of the E-waste isn't even getting collected and just thrown into the household waste. 80\% of the E-waste which got collected is then getting exported in poor countries. The recycling in these countries is problematic because E-waste contains lots of environmental contaminants and the facilities doesn't take proper care of this. This is why these contaminants are found around these premises. E-waste has already caused a "`considerable environmental degradation"'\cite{Robinson2009} in these countries. Also the workers are suffering from health problems because barely protected against the dangerous fluids and gasses. 
According to the current european WEEE-directive, manufacturers, sellers and distributors need to 
provide a return point for electronical and electrical devices. The aim is amongst others the reinforcement 
of recycling upon responsibility of the producer, which are also in charge of bearing the costs, while the 
end consumer has the responsibility of proper waste separation \cite{EURLEX}. Specifically for smartphones, 
the German Goverment rejects a deposit at the expense of the final consumer on the national implementation \cite{BMUB}. 
There are also existing several non-profit projects, which accept mobile phones in order to reuse and recycle them \cite{NABU, DUH}. 




\bibliography{bibliography}
\bibliographystyle{ACM-Reference-Format}

\end{document}
