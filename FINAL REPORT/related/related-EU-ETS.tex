The EU emissions trading system (EU ETS)\cite{watanabe2005european}, also known as the European Union Emissions Trading Scheme, 
was the first large greenhouse gas emissions trading scheme in the world, and now it's still the biggest one. 
The EU ETS works on the ``cap and trade'' principle. A cap is set on the total amount of certain greenhouse gases 
that can be emitted by installations covered by the system. The total emissions will fall if the cap is reduced 
over time. 

Under the ``cap and trade'' principle, a maximum (cap) is set on the total amount of greenhouse gases that can be 
emitted by all participating installations. ``Allowances'' for emissions are then auctioned off or allocated for 
free, and can subsequently be traded. Companies receive or buy emission allowances which they can trade with one 
another as needed. They can also buy limited amounts of international credits from emission-saving projects around 
the world. Trading brings flexibility that ensures emissions are cut where it costs least to do so. A robust carbon 
price also promotes investment in clean, low-carbon technologies.

Installations must monitor and report their CO2 emissions, ensuring they hand in enough allowances to the authorities 
to cover their emissions. If emission exceeds what is permitted by its allowances, an installation must purchase 
allowances from others.

The system covers the following sectors and gases with the focus on emissions that can be measured, reported 
and verified with a high level of accuracy: carbon dioxide (CO2), nitrous oxide (N2O), perfluorocarbons (PFCs).

According to the European Commission, in 2010 greenhouse gas emissions from big emitters covered by the EU ETS 
had decreased by an average of more than 17,000 tonnes per installation from 2005, a decrease of more than 8 percent since 2005.