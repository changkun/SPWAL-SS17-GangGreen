\begin{quotation}
  "\textit{Cumulatively, about 500 million PCs reached the end of their service lives between 1994 and 2003. 500 million PCs contain approximately 2,872,000 t of plastics, 718,000 t of lead, 1363 t of cadmium and 287 t of mercury}" \cite{global-perspective}\\
  
\end{quotation}

According to \cite{global-perspective} this already huge amount of e-waste is going to increase even further as electronics keep advancing and the need for new electronics keeps increasing. Exporting e-waste to poor countries would make sense for first world countries according to Larry Summers (back in 1991) since third world countries don't have an industry that already pollutes their air, water and ground so heavily, so they can deal with that problem more easily. Plus since mortality rates are already so high in these areas, the added pollution would not affect these countries that much.\\
This thinking started to change with the Basel Convention in 1989. It limits how much e-waste can be moved to what parts of the world, trying to save the environment and also trying to push the companies towards recycling. \\

Large household appliances and IT and telecommunications equipment made up three quarters of all e-waste back in 2002. This e-waste was mainly generated by countries of the OECD. Overall numbers are going to keep on increasing.\\

Although it is hard to track down how and where e-waste is going, there have to be put some restrictions into action on where e-waste is going. To solve the problem of this steadily increasing waste, recycling should become more important and more commonly used. But even for this solution, there are environmental hazards that come along with recycling.\\
To prevent e-waste from even becoming a problem, products should also be designed in such a way that they thrown away in a relatively short time span. 