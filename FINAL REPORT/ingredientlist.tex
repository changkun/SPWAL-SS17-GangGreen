\section{Ingredient List Specifications} 

In order to help determine how much e-waste is exactly being produced we want to introduce an ingredients list system which is similar to the EU legislation used for food products\cite{eu-ingredients-bmel}. Electronics manufacturer would be obligated to hand out a list with each electronic product to show how much and what kinds of raw material they use. Based on this a very precise set of rules and regulations can be created.\\
Hence, the high priority for us is to create an reasonable and concrete ingredients list with specific demands. 

\subsection{EU Regulations For Ingredients Lists}

According to \cite{eu-ingredients-info-pdf} there is a very specific set of rules that an ingredient list for food should be a aware of. For example, the ingredients should be ordered by weight. The more a certain ingredient was used for making the food, the further on top of the list it should stand. This makes the first item of the list become the most used ingredient and analogously the last item is the least used one. This makes it very easy to spot what the main ingredient is.\\

To increase transparency even further, the weights of some ingredients have to be listed very precisely. According to \cite{eu-ingredients-info-pdf}, if the manufacturer advertises certain ingredients on the packaging, he has to indicate how much of these advertised ingredients are contained percentagewise. Furthermore,  a set of certain ingredients always has to be listed with their respective weight portion of the product. Ingredients like sugar or fat falls into this category. \\
Special food additives not only have to be listed as an ingredient but their purpose also has to be made clear. In this way, the user knows why these unknown to him ingredients do. Food additives can also be listed as E-numbers.  \\

Further laws for these ingredients lists are in action but mentioning these is not necessary at this point. This excerpt should only give a rough overview about how ingredients have to be listed on products. 


\subsection{Our Ingredients List}

Our ingredients list for electronic devices will have a lot in common with those demanded by the European Union. Obviously there have to be some differences since we are dealing with electronic devices. \\

We will transfer the ordered list to our use case. There will be a list of all materials used in each electronic product. The very first material in this list will be the main ingredient and the last one will be the least used ingredient. Everything in between is ordered accordingly. \\
Special materials have to be brought to attention either by underlining them or making their font bold. Furthermore there has to be a very precise indication of how much each of these materials are being used by weight. Materials that fall into this category are either especially dangerous to the environment, rare materials or materials that are hard to recycle. \\

All other materials do not have to be indicated by weight. They will fall into the category of e-waste but are considered not to be especially dangerous to nature or humans. Their weight portion can be calculated by subtracting the weight of all of the specially indicated materials from the indicated total weight of the product itself. 

