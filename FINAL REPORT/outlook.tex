\section{Outlook}
As our proposed system of a ingredientlist for electronic equipment and devices is only a theoretical construct, as a first next step it is necessary to test it under real conditions. It is important to find out the practicability of the mapping of food related ingredientlists on electronical related ones. This largely depends on the application and has to be evaluated for every specific use case.\\
\\
In general, more application models can be developped considered, which also could require different specifications of the ingredientslist. For example the sorting of the list may be necessary to adjust: If the emphasize is on toxic materials, the order of the ingredients maybe should be oriented towards this criteria instead of the weight of all used materials, is the focus on special rare materials in order to boost their recycling and so save resources, it is more intuitive to give them a higher priority in the listing of ingredients. Also further other questions can appear: Is an extended labeling of the grade of toxicity necessary instead of just marking all toxic materials in the same way? What else has to be integrated on the ingredientlists (e.g. e-plastic)? Where are the limits of the system (potential need of labeling all components with a ingredientlist for partly recycling/resale)?\\
\\
It is also important to evaluate all effects of such a list as our proposed one. Potential influences on different several aspects has to be considered and analyzed one by one. For example there could be negative impacts on the economy: The list can be misconceived by end-consumers and lead to a significant decrease in consumption. Or the establishment of our ingredientlist could lead to a disadvantage of the european market against markets, which do not need to fulfill their requirements. Against that, our list could also rise the awareness of the public for the need to recycle. In general, all advantages of our system have to be evaluated in reality, so they can be weight up to negative side effects and disadvantages.\\
\\
We think, our proposed ingredientlist holds great potential. It provides a measuring system for e-waste, can raise general awareness in the consumers mind and so on. It provides the basis for further laws regarding the e-waste problem. This also includes to identify and distribute responsibilities to the different layers of EU government, organizations, producer and the consumer for both evaluating the ingredientlist itself as well as developing and executiing further rules based on the ingredientlist.\\
Is the latter once performed, then it is possible to draw comparisons between the resulting system and the e-waste managing systems mentioned in our related work section (see \ref{relatedwork}). Moreover, it is possible to think about ways, how these systems can be combined. Therefore is is neccessary to identify all benefits of the existing systems, which would have gone beyond the scope of this work.