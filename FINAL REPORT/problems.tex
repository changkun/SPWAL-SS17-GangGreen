\section{Challenges}
% How to get the precise numbers
At first, many producers of electronics may struggle to fulfill this law when their products contain parts delivered by other companies. It would take some time for all the producers to prepare the exact contents of their products. This may increase the price of the product, either because the producer has additional expenses for this or because they just might not do it if the market is too small.\\
% how precise should you be VS how precise can you be?
To be as precise as possible is very important for this approach, especially when the devices are very small and the components contain only grams or milligrams of certain declarable substances. The wanted precision might not always match a feasible one. The determination of the precision of the declaration could be very difficult and therefore has to be explored with care.\\
% what are the ingredients
A disadvantage of this system could concern importers who import non-labeled electronics which weren't originally designed for the European market. It would be the importers' duty to label it himself but mostly this is non-trivial or even impossible. The penalty the importer had to pay would increase the final price for the consumer which is bad for both the consumer and the economy. This problem will persist during the enrollment of this law but may fade with time when producers have caught up.\\
% cheating
One may argue also that it's quite complex to verify the lists the company\'s made and to control the implementing manufacturers. This is because of the volume of devices to control. This problem increases when financial benefits are tied to the contents of these lists although we have put forward a Game Theory. To work out a better solution to this is the task of the corresponding government.
The problems of this approach arise primarily from the context where it is implemented (see \ref{applications}).