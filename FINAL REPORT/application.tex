\section{Applications}

\subsection{Here is your stuff.}
...

\subsection{Ecosystem of Recycling}
%Maybe picture of this system, if space is required.
% source: Exploring e-waste management systems in the United States
Kahhat et al. proposed a "deposit-refund system"~\cite{•} for the successful collection and recycling of e-waste. In this economic system the consumer has to "pay a deposit on purchase, a variable portion of which is returned when turned in at the end-of-life"~\cite{•}. The interest collected by this deposit will compensate for overhead costs such as transportation and storage fees of recycling-companies. Moreover, as the returned deposit can vary, recycling companies with better recycling processes, can offer a higher return to consumers, as they will still gain revenue for the resources in recycled products. This system does not only ensure the return of e-waste by consumers, but also favors companies, who can recycle more effectively.

However, one huge downside of this system is the requirement to track to-be-sold devices, as the deposit is linked to the device. With our solution of an ingredient list for all released devices  this requirement could be avoided. We would eliminate the deposit (and with that the tracking) and collect additional tax, when purchasing electronics. Moreover, consumers will be able to sell their e-waste to recycling companies refunding their tax payment in the process. 

Another important advantage of the ingredient list is that, recycling companies know the contents of a device and as such could concentrate on specific recycling processes. Another weakness of Kahhats et al.'s system~\cite{•} is that, a company alone would not be able to recycle a product completely - however, recycling companies could negate this by selling partly recycled products to each other. The specialized companies, which would be possible by the ingredient list, would allow this selling process.

With the collected tax the government could support the establishment of recycling companies and in general the ecosystem of recycling dynamically. This offers huge flexibility, which is required in the area of electronics, as this market is a ever changing one. With the release of ingredient lists, the government can determine, which resources will be needed for production and  as a result for recycling. Thus, the government can subsidize effectively and visionary.

To sum up, the ingredient list could eliminate critical weaknesses of Kahhat et al.'s system, while also offering new possibilities.


\label{applications}