\section{Introduction}

Nowadays electronic devices around society has become a more and more crucial part of our lives. 
The demand for electronic devices increases and so does the amount of waste we have produced. 
E-waste is a term used for old electronic devices that are of no use anymore and thus becomes scrap. 
This causes severe pollution to our surroundings and other countries since most of the e-waste 
is being exported to third world countries.

In this paper, we proposed a solution that measures the E-waste of products which
compels the industries to release the \emph{ingredient} list of products under the regulation of government.
With this solution, the E-waste of each product $P$ has its indicator which is represented by 
the product features list or components list of e-waste:

\[
\text{E-waste}(P) = \sum_{i}{c_{i}}
\]

Where $c_{i}$ is the E-waste value of i-th ingredient component of the product $P$, 
which defined by industry standard. 

As a justification for this solution, we firstly 
investigate related works of current solutions and then establish the ingredient list as
a basic measurement to control the producing of the electronics. At the same time we put forward 
a scientific tax hierarchy and prevention mechanism to let the industries participate in this system forwardly.
Then we indicate that this study provides us an exciting opportunity to improve our E-waste recycling organizations.
Due to practical constraints, this paper cannot provide a comprehensive review of its 
rigorousness; the reader should bear in mind that the study is based on few assumptions that 
we interpreted in the subsequent section. Lastly, we also point out the outlooks of our solution.