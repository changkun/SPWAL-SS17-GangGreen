\section{Introduction}

Nowadays society electronics has become a more and more crucial part of our lives. 
The demand for electronic devices increases and so does the amount of waste we produce with them. 
E-waste is a term used for old electronic devices that are of no use anymore and thus become scrap. 
This causes severe pollution to our surroundings and other countries since most of the e-waste 
is being exported to third world countries.

In this paper, we proposed a solution that measures the E-waste of products through governments 
or organizations forces manufacturers list and evaluate products \emph{ingredient} E-waste.
With this solution, the E-waste of each product $P$ has its indicator which represented by 
the product features list or compoments list of e-waste:

\[
\text{E-waste}(P) = \sum_{i}{c_{i}}
\]

Where $c_{i}$ is the E-waste value of i-th ingredient component of the product $P$, 
which defined by industry standard. 

As a justification for this solution, we firstly 
investigate related works of current solutions and then established the ingredient list 
basic measurement requirements for the standard. Then we indicate that this study provides 
an exciting opportunity to advance our E-waste organizations and their regulations via 
two discussed example applications.
Due to practical constraints, this paper cannot provide a comprehensive review of its 
rigorousness; the reader should bear in mind that the study is based on few assumptions that 
we interpreted in the subsequent section. Lastly, we also point out the outlooks of our solution.