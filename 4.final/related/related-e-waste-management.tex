People are facing a hazardous problem in the modern era, more specifically saying, there are 20-50 million tons of 
e-waste being produced per year around the world, and this trend is still keeping going up. \cite{shah2015overview}

The e-waste originally appeared in the developed countries, but now expanded to the developing countries, even less 
developed countries, which brings them both advantage and disadvantage.

The reason why e-waste has such contradictory traits mainly lies on the two aspects of e-waste, which are supposed to 
be treated respectively:

\begin{itemize}
	\item Poisonous parts in the e-waste.(disadvantages)
	\item Valuable parts in the e-waste.(advantages)
\end{itemize}

Therefore, how to maximize the advantages, meanwhile minimize the disadvantage becomes a priority for those suffering 
countries.

Traditional ways to deal with e-waste will consist of four procedures: Recycling, Landfills, Composition of e-waste, 
and Security issues.
The paper also provides several approaches: Life Cycle Assessment, Material Flow Analysis, Multi-Criteria Analysis , 
Extended Producer Responsibility, Health Implications.

A good management system should apply above approaches into procedures. How can we innovate over the traditional method
and put forward a more scientific and efficient way could be a interesting proposal for our final project.