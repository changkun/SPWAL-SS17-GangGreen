\documentclass[sigchi-a, authorversion]{acmart}
\usepackage{booktabs} % For formal tables
\usepackage{ccicons}  % For Creative Commons citation icons
\usepackage{url}
\usepackage{hyperref}


% Information
\acmConference[SPWAL SS17]{Seminar und Praktikum "Wissenschaftliches Arbeiten und Lehren" Sommersemester 2017}{\today}{Munich, Germany} 

\begin{document} 
\title{Progress Report Phase 1: A system to regulate E-waste export in the EU}

% Author
\author{Anita Baier}
\affiliation{\institution{Ludwigs-Maximilian-Universit\"at}
  \city{Munich} \country{Germany}}
\email{anita.baier@gmx.de} 

\author{Zhe Li}
\affiliation{\institution{Ludwigs-Maximilian-Universit\"at}
  \city{Munich} \country{Germany}}
\email{li.zhe@campus.lmu.de} 

\author{Jonas Mattes}
\affiliation{\institution{Ludwigs-Maximilian-Universit\"at}
  \city{Munich} \country{Germany}}
\email{j.mattes@campus.lmu.de} 

\author{Bruno M\"uller}
\affiliation{\institution{Ludwigs-Maximilian-Universit\"at}
  \city{Munich} \country{Germany}}
\email{muellerbr@cip.ifi.lmu.de} 

\author{An Tien Go}
\affiliation{\institution{Ludwigs-Maximilian-Universit\"at}
  \city{Munich} \country{Germany}}
\email{antiengo@campus.lmu.de}

\author{Changkum Ou}
\affiliation{\institution{Ludwigs-Maximilian-Universit\"at}
  \city{Munich} \country{Germany}}
\email{hi@changkun.us} 

\author{Guoliang Xue}
\affiliation{\institution{Ludwigs-Maximilian-Universit\"at}
  \city{Munich} \country{Germany}}
\email{guoliang.xue@campus.lmu.de} 


\ccsdesc[500]{Hardware~Power and energy}
\ccsdesc[300]{Hardware~Impact on the environment}


\begin{abstract}

  In todays society electronics become a more and more crucial part of our lives. 
  The demand for electronic devices increases and so does the amount of waste we produce with them. 
  E-waste is a term used for old electronic devices that are of no use anymore and thus become scrap. 
  This causes serious pollution to our surroundings and other countries since most of the e-waste is 
  being exported to third world countries.
  
  Our goal is to find a solution to this pollution problem. Our plan is to create a trading system similar 
  to the CO2 trading system of the European Union. Countries are allowed to only export a certain amount of e-waste. 
  This amount can be increased if they either buy allowances from other countries or if they invest in recycling of e-waste. 
  The global e-waste export limits will be lowered year by year, thus lowering global pollution created by no longer
  used electronics.  
  
\end{abstract}

\keywords{E-waste; Fee-Upon-Disposal Regulation; recycling.}

\maketitle

% \begin{sidebar}
%   \textbf{Good Utilization of the Side Bar} 
  
%   \textbf{Preparation:} Do not change the margin
%   dimensions and do not flow the margin text to the
%   next page. 
  
%   \textbf{Materials:} The margin box must not intrude
%   or overflow into the header or the footer, or the gutter space
%   between the margin paragraph and the main left column. 
  
%   \textbf{Images \& Figures:} Practically anything
%   can be put in the margin if it fits. Use the
%   \texttt{{\textbackslash}marginparwidth} constant to set the
%   width of the figure, table, minipage, or whatever you are trying
%   to fit in this skinny space.

%   \caption{This is the optional caption}
%   \label{bar:sidebar}
% \end{sidebar}

%\begin{figure}
%  \includegraphics[width=\marginparwidth]{sigchi-logo}
%  \caption{Insert a caption below each figure.}
%  \label{fig:sample}
%\end{figure}


\section{Introduction}
E-waste is a growing problem, especially in industrialized countries. One of the applicated solutions
is to ship this e-waste to third world countries. However, this cannot be the longterm solution. As such
we propose an e-waste allowance system. With this, only a limited amount of e-waste can be shipped outside, 
which will be reduced yearly.

\section{Related Work}

TBA

\section{Justification}

Many solution to the e-waste problems are extreme (as seen in the related work section). However, it is important
for a successful transfer from old plans to efficient e-waste plans to make small adjustments first. 
For this reason, we introduce an e-waste allowance. This will force countries to slowly but surely reduce their e-waste production.

\section{Evaluation}

In order to check the vadility of this approach, we will introduce this system first for specific device. 
For example, only TVs should be limited in exporting. With the results of the testing phase, we will 
re-evaluate our system.

\section{Research Plan}

Our system will be similar to the C02 system of the EU. As such, we will analyze the C02 plan and re-adjust 
it to our needs. For this, these important aspects have to be considered: the measurement of e-waste, 
the distribution of e-waste-shipping-certificates (similar to the C02-certificates) and how many certificates 
should be available. 
In the first progress report, we will present a draft-system for this. In the mid term syncrhonisation we expect
to have the system more fleshed out. In the second report we expect to have the system completely worked out.
In the final stage, we will expand our research question in other ways. For example, with our proposed idea no clear
progress can be measured (when considering everything), as only returned e-waste is monitored. One solution to this would
be to research how e-waste could be tracked. 

\bibliography{bib}
\bibliographystyle{bibformat}

\end{document}
