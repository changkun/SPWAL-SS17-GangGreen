\documentclass[sigchi-a, authorversion]{acmart}
\usepackage{booktabs} % For formal tables
\usepackage{ccicons}  % For Creative Commons citation icons
\usepackage{url}
\usepackage{hyperref}


% Information
\acmConference[SPWAL SS17]{Seminar und Praktikum "Wissenschaftliches Arbeiten und Lehren" Sommersemester 2017}{\today}{Munich, Germany} 

\begin{document} 
\title{Progress Report Phase 1: A system to regulate E-waste export in the EU}

% Author
\author{Anita Baier}
%\affiliation{\institution{Ludwigs-Maximilian-Universit\"at}
%  \city{Munich} \country{Germany}}
%\email{anita.baier@gmx.de} 

\author{Zhe Li}
%\affiliation{\institution{Ludwigs-Maximilian-Universit\"at}
%  \city{Munich} \country{Germany}}
%\email{li.zhe@campus.lmu.de} 

\author{Jonas Mattes}
%\affiliation{\institution{Ludwigs-Maximilian-Universit\"at}
%  \city{Munich} \country{Germany}}
%\email{j.mattes@campus.lmu.de} 

\author{Bruno M\"uller}
%\affiliation{\institution{Ludwigs-Maximilian-Universit\"at}
%  \city{Munich} \country{Germany}}
%\email{muellerbr@cip.ifi.lmu.de} 

\author{An Tien Go}
%\affiliation{\institution{Ludwigs-Maximilian-Universit\"at}
%  \city{Munich} \country{Germany}}
%\email{antiengo@campus.lmu.de}

\author{Changkum Ou}
%\affiliation{\institution{Ludwigs-Maximilian-Universit\"at}
%  \city{Munich} \country{Germany}}
%\email{hi@changkun.us} 

\author{Guoliang Xue}
%\affiliation{\institution{Ludwigs-Maximilian-Universit\"at}
%  \city{Munich} \country{Germany}}
%\email{guoliang.xue@campus.lmu.de} 


\ccsdesc[500]{Hardware~Power and energy}
\ccsdesc[300]{Hardware~Impact on the environment}


\begin{abstract}

  In todays society electronics become a more and more crucial part of our lives. 
  The demand for electronic devices increases and so does the amount of waste we produce with them. 
  E-waste is a term used for old electronic devices that are of no use anymore and thus become scrap. 
  This causes serious pollution to our surroundings and other countries since most of the e-waste is 
  being exported to third world countries.
  
  Our goal is to find a solution to this pollution problem. Our plan is to create a trading system similar 
  to the CO2 trading system of the European Union. Countries are allowed to only export a certain amount of e-waste. 
  This amount can be increased if they either buy allowances from other countries or if they invest in recycling of e-waste. 
  Meanwhile, we leverage a mathematical model to prove our system's rationality.
  The global e-waste export limits will be lowered year by year, thus lowering global pollution created by no longer
  used electronics.  
  
\end{abstract}

\keywords{E-waste; Fee-Upon-Disposal Regulation; recycling.}

\maketitle

% \begin{sidebar}
%   \textbf{Good Utilization of the Side Bar} 
  
%   \textbf{Preparation:} Do not change the margin
%   dimensions and do not flow the margin text to the
%   next page. 
  
%   \textbf{Materials:} The margin box must not intrude
%   or overflow into the header or the footer, or the gutter space
%   between the margin paragraph and the main left column. 
  
%   \textbf{Images \& Figures:} Practically anything
%   can be put in the margin if it fits. Use the
%   \texttt{{\textbackslash}marginparwidth} constant to set the
%   width of the figure, table, minipage, or whatever you are trying
%   to fit in this skinny space.

%   \caption{This is the optional caption}
%   \label{bar:sidebar}
% \end{sidebar}

%\begin{figure}
%  \includegraphics[width=\marginparwidth]{sigchi-logo}
%  \caption{Insert a caption below each figure.}
%  \label{fig:sample}
%\end{figure}


\section{Achievements}

We collect and analyze related data. Several diagrams would be plotted to demonstrate what the current situation of E-waste in EU is, and which hazards it has brought into citizens’ daily life. More specifically, we could find out what the development tendency of E-waste is in respective country, and is there any difference in terms of volume of E-waste among countries. 
So now we have a better understanding about the current situation of E-waste: keeping increasing E-waste, rigorous solutions(eg. ship e-waste to third world countries)\cite{Robinson2009}, ineffective solutions (eg. deposit for producer), recycling and reusing only not long-lasting lower e-waste\cite{kahhat2008exploring,BMUB}.


\section{Next Step}

\subsection{More Researches about Measurement}

We have done some researches about how the existing methods \cite{DUH,NABU} contend with E-waste problem, and find out whether they have any flaws that we can avoid in my project, or could we put forward a new way to solve those issues. Just as the proposal describes, CO2 trading system will be an appropriate analogy to our system. We should leverage the \" cap and trade \" principle to solve the reasonable distribution of E-waste. In the next step, we accentuate the task how to measure the E-waste and what is the quality criteria.

\subsection{System Design}

To provide a better prerequisite for mathematical model, a robust system design is necessary. We could use more scientific and efficient way \cite{shah2015overview} to manage system. Additionally, from the regulation aspect, Plambeck and Wang \cite{plambeck2009effects} investigates the influence of how e-waste regulation types and market structure affects related four aspects of new production introduction: the quantify of e-waste, manufacturers’ profits, consumer surplus and social welfare. So we also need to integrate those four aspects to our system.

\subsection{Mathematical Proof}

Since many solution to the e-waste problems are extreme, it is important for a successful transfer from old plans to efficient e-waste plans to make small adjustments first. 
For this reason, we introduce an e-waste allowance\cite{watanabe2005european}. This will force countries to slowly but surely reduce their e-waste production. Therefore, we try to build a mathematical model \cite{global-perspective}, which could efficiently solve the allocation of E-waste all around countries in EU. Optimized resource allocation algorithm will play a important role to implement this model.  After that we try to prove its rationality. This part is the most crucial part.

Here we introduce a new concept \" certificate \" as an impact factor. we could impose the restriction on the distribution of e-waste-shipping by certificate, at the same time,  the number of available certificates for each country is supposed to consider as well.

\subsection{Additional Aspects}

In the following, it is necessary to mention what the risks our model is going to face and present our limitations of our project. Never to ignore is to analyze the reasons behind those risks.
In the last section a conclusion and future work on our topic will be provided.

\bibliography{bib}
\bibliographystyle{bibformat}

\end{document}
