From the regulation aspect, Plambeck and Wang  \cite{plambeck2009}  investigates the influence of how e-waste regulation types (Fee Upon Sale and Fee Upon Disposal) and market structure (Duopoly and Monopoly) affects related four aspects of new production indtroduction: the quantify of e-waste, manufacturers' profits, consumer surplus and social welfare. 
As instance, they considered the California's Advanced Recovery Fee (ARF) which is a fee-upon-sale strategy that forces customer pay for the collection and recycling of all used electronics when they buy an new product. For analysis, their models explained the fee-upon-disposal extended producer responsibility motivate design for recyclability which is benefits for consumers but fail to reduce the frequency of new products launching.
In consequences, optimally induces electronics manufactures to both slow research/development and design benign products is the further challenge for our future form invention of e-waste regulation.